\documentclass{beamer}
\usepackage{listings}
\usepackage{minted}
\usepackage{anyfontsize}
\usepackage[utf8]{inputenc}

\mode<presentation> {
  \usetheme{Madrid}
}

\title[Selenium Page Object Model]{Page Object Model} % The short title appears at the bottom of every slide, the full title is only on the title page

\author{Radu Ciorba \href{mailto:radu@presslabs.com}{\{radu@presslabs.com\}}} % Your name
\date{\today} % Date, can be changed to a custom date

\begin{document}

\begin{frame}
  \titlepage % Print the title page as the first slide
\end{frame}

\begin{frame}
  \frametitle{Content of this presentation}
  \begin{itemize}
  \item Introducing the Page Object Model pattern
  \item Selenium Tips and Tricks
  \end{itemize}
\end{frame}

\begin{frame}
  \frametitle{Problems: Flaky Tests}
  \begin{itemize}
  \item Flaky Tests
  \item Mostly caused by timing issues
  \item sleep won't fix this
  \end{itemize}
\end{frame}

\begin{frame}
  \frametitle{Flakyness}
  \inputminted[fontsize=\footnotesize]{python}{snippets/waits.py}

\end{frame}

\begin{frame}
  \frametitle{Problems: Readability}
   {\fontsize{8.5pt}{8.5pt}
     \inputminted{python}{snippets/before.py}
   }
\end{frame}

\begin{frame}
  \frametitle{Problems: Readability}
   {\fontsize{8.5pt}{8.5pt}
     \inputminted{python}{snippets/after.py}
   }
\end{frame}

\begin{frame}
  \frametitle{Problems: Readability}
  \begin{itemize}
  \item For the test code \textbf{Intent} should be obvious
  \item Don't lose \textbf{the what} in a forest of \textbf{how}
  \item Add another layer of abstraction, to separate the details of the DOM from the test code
  \end{itemize}
\end{frame}

\begin{frame}
  \frametitle{A Simple Page Object}
  \inputminted[fontsize=\footnotesize]{python}{snippets/pom1.py}
\end{frame}

\begin{frame}
  \frametitle{A Page Object with Nested Elements}
  \inputminted[fontsize=\footnotesize]{python}{snippets/pom2.py}
\end{frame}

\begin{frame}
  \frametitle{Separation of concernes}
  What belongs in the POM library
  \begin{itemize}
  \item Generic code, that's probably gonna be the same to interact with any page, element
  \item click, hover, input text, read text, etc.
  \end{itemize}
  \vspace{10pt}
  What belongs in the Pages
  \begin{itemize}
  \item What elements are on this page
  \item getters for reading certain attributes (coment author's email, etc)
  \end{itemize}
  \vspace{10pt}
  What belongs in the Tests
  \begin{itemize}
  \item What actions are being performed (if the DOM structure is visible here, you have an abstraction leak)
  \end{itemize}
\end{frame}

\begin{frame}
  \frametitle{Tips and tricks}
  \begin{itemize}
  \item Try to keep tests independent (not always trivial, especially since we're black box testing someone else's code)
  \item If there are dependencies, try to snapshot the DB after reaching a good state, then revert there (see the must-use plugin in the rctest instance)
  \end{itemize}
\end{frame}

\begin{frame}[fragile]
  \frametitle{Tips and tricks: Nested X server}
  \begin{itemize}
  \item Xephyr with Chrome or FFox
  \item Avoids anoying pop up windows, but you can still see/interact
  \end{itemize}
  \begin{verbatim}
    Xephyr -ac -screen 1918x1058 :99 &
    export DISPLAY=:99
\end{verbatim}
\end{frame}

\begin{frame}[fragile]
  \frametitle{Tips and tricks: Drop to iPdb shell on error}
  \begin{itemize}
  \item Keeps browser open, and allows you to debug
  \end{itemize}
  \begin{verbatim}
    py.test --capture=no
            --pdb
            --pdbcls=IPython.terminal.debugger:Pdb
\end{verbatim}
\end{frame}

\begin{frame}[fragile]
  \frametitle{Tips and tricks: Snapshot DB and just re-run one test}
  \begin{itemize}
  \end{itemize}
  \begin{verbatim}
from utils.snapshot import take_snapshot, restore_snapshot
def test_approve_comment():
    # test code here
    take_snapshot("rc")
    
def test_reply_comment():
    restore_snapshot("rc")
    # test code here


$ py.test tests/test_comments.py::test_reply_comment
    
\end{verbatim}
\end{frame}

\end{document}
