\documentclass{beamer}
\usepackage{listings}
\usepackage{anyfontsize}
\usepackage[cache=false,outputdir=.texpadtmp]{minted}


\mode<presentation> {
  \usetheme{Madrid}
}

\title[Async Python]{What is this async def and why is it aswesome} % The short title appears at the bottom of every slide, the full title is only on the title page

\author{Radu Ciorba \href{mailto:radu@devrandom.ro}{\{radu@devrandom.ro\}}} % Your name
%% \institute[devrandom.ro] % Your institution as it will appear on the bottom of every slide, may be shorthand to save space
%% {
%%   %% Radu Ciorba \\ % Your institution for the title page
%%   \medskip
%%   \textit{radu.ciorba@bitdefender.com} % Your email address
%% }
\date{\today} % Date, can be changed to a custom date

\begin{document}

\begin{frame}
  \titlepage % Print the title page as the first slide
\end{frame}

\begin{frame}
  \frametitle{What is the problem we're trying to solve?}
  \begin{itemize}
  \item Concurrency?
  \item Make Python WebScale?
  \end{itemize}
\end{frame}

\begin{frame}
  \frametitle{CPU Bound vs IO Bound}
  \begin{itemize}
  \item Mine Crypto-Currency - CPU heavy workload
  \end{itemize}
\end{frame}

\begin{frame}
  \frametitle{CPU Bound vs IO Bound}
  A common use case: a python web application
  \begin{itemize}
  \item 1 Listen for incomming connections
  \item 2 Accept a connection
  \item 3 Read the request data from the connection
  \item \colorbox{yellow!30}{4 Parse the request}
  \item \colorbox{yellow!30}{5 Call the request handler}
  \item 6 Request handler queries the DB a few times
  \item \colorbox{yellow!30}{7 Render a template using the data we got in step 6}
  \item 8 Write the response out to the connection
  \end{itemize}
\end{frame}

\begin{frame}
  \frametitle{Different approaches to concurrency}
  Threads (very similar for Processes)
  \begin{itemize}
  \item \colorbox{yellow!30}{1 Listen for incomming connections}
  \item \colorbox{yellow!30}{2 Accept a connection}
  \item \colorbox{yellow!30}{ * Start a new Thread, pass it the connection we just accepted}
  \item 3 Read the request data from the connection
  \item 4 Parse the request
  \item 5 Call the request handler
  \item 6 Request handler queries the DB a few times
  \item 7 Render a template using the data we got in step 6
  \item 8 Write the response out to the connection
  \end{itemize}
\end{frame}

\begin{frame}[fragile]
  \frametitle{An Async Example}
  \inputminted[fontsize=\footnotesize]{javascript}{code/async.js}
  %% \verbatimfont{\footnotesize}
  %% \begin{verbatim}
  %% $(document).on('submit', 'form.remember', function(){
  %%    // code
  %% });
  %% \end{verbatim}
\end{frame}


\begin{frame}
  Also known as: Callback Hell
 \begin{center}
 \includegraphics[scale=0.20,keepaspectratio=true]{./thisisfine.jpeg}
 \end{center}
\end{frame}

\begin{frame}[fragile]
  \frametitle{A Python Tornado Async Example}
  \inputminted[fontsize=\footnotesize]{python}{code/tornado1.py}
  %% \verbatimfont{\footnotesize}
  %% \begin{verbatim}
  %% $(document).on('submit', 'form.remember', function(){
  %%    // code
  %% });
  %% \end{verbatim}
\end{frame}

\begin{frame}
  \frametitle{Async in Python}
  \begin{itemize}
    \item 2002 - Twisted
    \item 2006 - Eventlet
    \item 2009 - Tornado
    \item 2009 - Gevent
    \item 2011 - PEP 3153 -- Asynchronous IO support
    \item 2012 - Tulip \& PEP 3156 -- Asynchronous IO Support Rebooted
    \item 2013 - Tulip becomes AsyncIO in 3.4
  \end{itemize}
\end{frame}

\end{document}
