\documentclass{beamer}
\usepackage{listings}
\usepackage{anyfontsize}
\usepackage[cache=false,outputdir=.texpadtmp]{minted}


\mode<presentation> {
  \usetheme{Madrid}
}

\title[Async Python]{What is this async def and why is it aswesome} % The short title appears at the bottom of every slide, the full title is only on the title page

\author{Radu Ciorba \href{mailto:radu@devrandom.ro}{\{radu@devrandom.ro\}}} % Your name
%% \institute[devrandom.ro] % Your institution as it will appear on the bottom of every slide, may be shorthand to save space
%% {
%%   %% Radu Ciorba \\ % Your institution for the title page
%%   \medskip
%%   \textit{radu.ciorba@bitdefender.com} % Your email address
%% }
\date{\today} % Date, can be changed to a custom date

\begin{document}

\begin{frame}
  \titlepage % Print the title page as the first slide
\end{frame}

\begin{frame}[fragile]
  \frametitle{async def hello\_world()}
  \inputminted[fontsize=\footnotesize]{python}{code/hw.py}
\end{frame}

\begin{frame}
  \frametitle{What is the problem we're trying to solve?}
  \begin{itemize}
  \item Do stuff concurrently
  \end{itemize}
\end{frame}

\begin{frame}
  \frametitle{What is the problem we're trying to solve?}
  \begin{center}
  \includegraphics[scale=0.70,keepaspectratio=true]{./python-web-scale.jpeg}
  \end{center}
\end{frame}

\begin{frame}
  \frametitle{CPU Bound vs IO Bound}
  \begin{itemize}
  \item Mine Crypto-Currency - CPU heavy workload
  \end{itemize}
\end{frame}

\begin{frame}
  \frametitle{CPU Bound vs IO Bound}
  A common use case: a python web application
  \begin{itemize}
  \item 1 Listen for incomming connections
  \item 2 Accept a connection
  \item 3 Read the request data from the connection
  \item \colorbox{yellow!30}{4 Parse the request}
  \item \colorbox{yellow!30}{5 Call the request handler}
  \item 6 Request handler queries the DB a few times
  \item \colorbox{yellow!30}{7 Render a template using the data we got in step 6}
  \item 8 Write the response out to the connection
  \item Rinse, repeat.
  \end{itemize}
\end{frame}

\begin{frame}
  \frametitle{Different approaches}
  \begin{itemize}
  \item Threads
  \item Processes
  \item non blocking IO + event loop
  \end{itemize}
\end{frame}

\begin{frame}
  \frametitle{Threads}
  Threads add overhead:
  \begin{itemize}
  \item Each thread has it's own stack
  \item Context switching
  \item Shared state - easy to introduce bugs
  \item The GIL - not a very big deal for IO bound problems
  \item Still - good enough in many cases
  \end{itemize}
\end{frame}

\begin{frame}
  \frametitle{Event loop}
  \begin{itemize}
  \item Single thread of execution
  \item Hand your code over to the EL to run it
  \item Don't do your own IO, ask the EL to do it for you
  \item Basically a form of cooperative concurrency
  \item No really, don't do your own IO, cause you'll block the EL
  \end{itemize}
\end{frame}

%% \begin{frame}
%%   \frametitle{The same webapp example, with Threads}
%%   Threads (very similar for Processes)
%%   \begin{itemize}
%%   \item \colorbox{yellow!30}{1 Listen for incomming connections}
%%   \item \colorbox{yellow!30}{2 Accept a connection}
%%   \item \colorbox{yellow!30}{ * Start a new Thread, pass it the connection we just accepted}
%%   \item 3 Read the request data from the connection
%%   \item 4 Parse the request
%%   \item 5 Call the request handler
%%   \item 6 Request handler queries the DB a few times
%%   \item 7 Render a template using the data we got in step 6
%%   \item 8 Write the response out to the connection
%%   \end{itemize}
%% \end{frame}

\begin{frame}[fragile]
  \frametitle{An Async Example}
  \inputminted[fontsize=\footnotesize]{javascript}{code/async.js}
  %% \verbatimfont{\footnotesize}
  %% \begin{verbatim}
  %% $(document).on('submit', 'form.remember', function(){
  %%    // code
  %% });
  %% \end{verbatim}
\end{frame}


\begin{frame}
  Also known as: Callback Hell
 \begin{center}
 \includegraphics[scale=0.20,keepaspectratio=true]{./thisisfine.jpeg}
 \end{center}
\end{frame}

\begin{frame}[fragile]
  \frametitle{A Python Tornado Async Example}
  \inputminted[fontsize=\footnotesize]{python}{code/tornado1.py}
  %% \verbatimfont{\footnotesize}
  %% \begin{verbatim}
  %% $(document).on('submit', 'form.remember', function(){
  %%    // code
  %% });
  %% \end{verbatim}
\end{frame}

\begin{frame}
  \frametitle{Async in Python}
  \begin{itemize}
    \item 1999 - asyncore lands in the stdlib for 1.5.2
    \item 2002 - Twisted
    \item 2009 - Tornado
    \item 2009 - Gevent
    \item 2011 - PEP 3153 -- Asynchronous IO support
    \item 2012 - Tulip \& PEP 3156 -- Asynchronous IO Support Rebooted
    \item 2013 - Tulip becomes AsyncIO in 3.4
  \end{itemize}
\end{frame}

\begin{frame}
  \frametitle{AsyncIO goals}
  \begin{itemize}
    \item Provide an API for event loops, allow interop between projects
    \item Offer a solution to avoid callbacks (not required)
    \item Work on python 3.4 as it was then, without changes to the language
  \end{itemize}
\end{frame}

\begin{frame}
  \frametitle{Async without callbacks: Coroutines}
  \begin{itemize}
    \item Provide an API for event loops, allow interop between projects
    \item Coroutines: a solution to avoid callbacks (not required)
    \item Work on python 3.4 as it was then, without changes to the language
  \end{itemize}
\end{frame}

\begin{frame}[fragile]
  \frametitle{Coroutines without changin the language?}
  \inputminted[fontsize=\footnotesize]{python}{code/coro.py}
\end{frame}

\begin{frame}[fragile]
  \frametitle{New language features: async def/await}
  \inputminted[fontsize=\footnotesize]{python}{code/async1.py}
\end{frame}


\begin{frame}[fragile]
  \frametitle{New language features: async for}
  \inputminted[fontsize=\footnotesize]{python}{code/async2.py}
\end{frame}

\begin{frame}[fragile]
  \frametitle{New language features: async with}
  \inputminted[fontsize=\footnotesize]{python}{code/async3.py}
\end{frame}

\begin{frame}[fragile]
  \frametitle{New language features: yield in coroutines}
  \inputminted[fontsize=\footnotesize]{python}{code/async4.py}
\end{frame}

\begin{frame}
  \frametitle{Library support}
  \begin{itemize}
    \item Sanic - Flask like web framework
    \item Motor - mongo driver, originally for tornado
    \item elasticsearcy-py-async - for ES
    \item aio-pika - for rabbitmq
    \item \href{https://github.com/aio-libs}{aio-libs project} - aiomysql, aiopg, aioredis, aiohttp
  \end{itemize}
\end{frame}

\begin{frame}[fragile]
  \frametitle{Sanic + Motor}
  \inputminted[fontsize=\footnotesize]{python}{code/sanic.py}
\end{frame}

\begin{frame}
  \frametitle{Where we are today}
  \begin{itemize}
    \item Elegant way to write async code in Python
    \item A growing echosystem of asynio compattible libs
  \end{itemize}
\end{frame}


\begin{frame}
  \frametitle{Alternatives}
  Gevent
  \begin{itemize}
    \item uses the greenlets package
    \item write code in blissful ignorance of the Event Loop
    \item gevent.monkey.patch\_all() - replaces blocking IO calls with cooperative versions
    \item can't monkey patch C extensions - use pure python libs
    \item many 3rd party libs will work (requests, django, any pure python code should work)
    \item io in C extesions can still block the event loop
    \item unlike async/await your code can be suspended/resumed at any time (like threads)
  \end{itemize}
\end{frame}

\end{document}
